\chapter{Theoretical Background}\label{chapter:background}
\thispagestyle{chapterBeginStyle}

\begin{quote}{Bishop}
Thomas Bayes was born in Tunbridge Wells and was a clergyman as well as an amateur scientist and a mathematician. He studied logic and theology at Edinburgh University and was elected Fellow of the Royal Society in 1742. During the 18th century, issues regarding probability arose in connection with
gambling and with the new concept of insurance. One particularly important problem concerned so-called inverse probability. A solution was proposed by Thomas Bayes in his paper ‘Essay towards solving a problem in the doctrine of chances’, which was published in 1764, some three years after his death, in the Philosophical Transactions of the Royal Society. In fact, Bayes only formulated his theory for the case of a uniform prior, and it was Pierre-Simon Laplace who independently rediscovered the theory in general form and who demonstrated its broad applicability. \cite{Bishop2006}
\end{quote}

\section{First section}
\subsection{First subsection}
\subsection{Second subsection}

\section{Second section}
\subsection{First subsection}
\subsection{Second subsection}